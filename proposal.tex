\documentclass[a4paper,11pt]{article}

\usepackage[utf8]{inputenc}

\usepackage{natbib}

\usepackage{hyperref}

\hypersetup{
  colorlinks   = true, %Colours links instead of ugly boxes
  urlcolor     = blue, %Colour for external hyperlinks
  linkcolor    = blue, %Colour of internal links
  citecolor   = black %Colour of citations
}

\bibliographystyle{apalike}

\title{Proposal for the Second Edition of \\
  Event History Analysis with R\footnote{Second attempt.}}


\author{Göran Broström}

\begin{document}

\maketitle

\tableofcontents

\section{Introduction}

I have been asked by Chapman \& Hall to write a second edition of my book
\emph{Event History Analysis with R} \citep{ehar12}. I have accepted, and
this is a proposal for the second edition. It includes both updates as a
consequence of the development of the companion {\bf R} \citep{cran}
package {\tt eha} \citep{eha} and new material. A general description of
news is given in Section~\ref{sec:news} and suggested updates are given in
Section~\ref{sec:updates}.

Finally, in Section~\ref{sec:contents}, a tentative layout of Chapters and
Sections in the new edition is given.

This is a revised version of the original proposal, based on input from reviewers 
(for which I am grateful).

\section{News} \label{sec:news}

\subsection{Reproducible research and RStudio}

The importance of \emph{reproducible research} is recognized today
\citep{gandrud15,vsflrp}. The topic will be introduced,  tools for it
recommended, and they will be used throughout the book. This includes using 
RStudio \citep{rstudio} and Rmarkdown \citep{rmarkdown}.

\subsection{Register-based data methods}

I am working (well, professor emeritus) at {\sc CEDAR}, Umeå University,
Umeå, Sweden. {\sc Cedar} 
stands for  \emph{Centre for Demographic and Ageing Research} and it is a
centre for both research and data base building and keeping. One of the
largest data bases at {\sc Cedar} is the \emph{Linn{\ae}us data base},
consisting of yearly individual information about the population of Sweden
between the years 1986 and 2013. Sweden has about 9 million inhabitants
today, so this data base is huge.

Several research projects utilizing the \emph{Linn{\ae}us} data base are
applying event history analysis one way or the other. They regularly
encounter two typical problems:

\begin{enumerate}

\item Data are discrete in nature (observation of state one day per year),
  so performing \emph{Cox regression} is not straightforward.
\item The huge amount of data makes standard survival analysis programs
  choke.
\end{enumerate}  

I intend to show how to handle these aspects with register-based data. The
R package eha \citep{eha} will be developed accordingly.

\subsection{Reducing huge data sets by statistical sufficiency}

The sufficiency principle can sometimes be used to reduce the size of a data set 
to a fixed-dimension data set (a table). This will speed up compilation considerably
given that certain conditions are met. This is something to implement in the eha
package in the first place, but a description of how it works will be included 
in the book as well.

\subsection{The weird bootstrap}

\emph{The weird bootstrap} \citep[p. 323]{abgk93} is implemented in the package eha, but 
not explained in the first edition of the book. It will now be introduced.

\subsection{Presentation}

Presentation refers to how to present the results of a statistical
investigation, for instance of a Cox regression. My view is that graphical
presentations should be promoted and tables kept to a minimum, so I will
investigate how certain kinds of table content can be translated into a graph.

\subsubsection{The Hauck-Donner effect, $p$-values, and contrasts}

Traditional ways of presenting the results a Cox or AFT regression (or any 
regression) are neglecting the multiple testing problem, which is especially 
serious in connection with contrasts for categorical covariates. I will show 
the correct way(s) of doing things.

See \url{http://capa.ddb.umu.se/ds/contrasts.html} for a hint of what I have in 
mind here.

\subsubsection{Tables}

The issue is what information should go into a table: I am especially
critical to the way $p$-values are presented in standard layouts.  One
interpretation of a $p$-value is that it is just another statistic,
computable from data. That is fine, but problem arises when the $p$-value is
interpreted as a probability and statements about ``significance'' are made.

\subsection{Cox regression}

This is more about a changing focus of the eha package: The non-parametric
approach (Cox regression) is well covered by the survival package
\citep{survival}, and the corresponding parts of the eha package will be
toned down. Instead, things that eha can do but survival cannot will be emphasized.

However, this is more about changes in the package eha, and it will not change the
presentation in the book significantly.



This will also be reflected in the second edition of the book. 

\subsection{Parametric survival models}


\subsubsection{The Gompertz baseline distribution}

The Gompertz distribution is traditionally used in modeling human adult
mortality, by good reasons. However, a ``common knowledge'' that \emph{the
Gompertz distribution does not lend itself to be used in AFT modeling} 
\citep[p. 285]{klekle} is
\emph{not true}. On the contrary, the Gompertz distribution can be used both in PH
and AFT modeling. This will be explained in the book, and the benefits from
this will be pointed out. It includes the possibility to perform mediation
analysis in a simple way because the AFT model is log-linear.


\subsubsection{The Piecewise Constant Hazard model}

This parametric model is close to a non-parametric approach (let the sizes
of the pieces tend to zero \ldots) and is an excellent replacement for Cox
regression with massive data sets: With categorical covariates data can be
reduced to fixed-dimensional tables by sufficiency and the equivalence
with Poisson regression can be utilized for very fast and accurate
estimation. This will be implemented ih the eha package and described in
the book. It should be noted that this is already possible to do with base R
and some programming, ant that is what will be implemented and described.   

\subsubsection{Mediation analysis}

Ways of performing mediation analysis in parametric survival models are
given, especially in the accelerated failure time framework.


\section{Updates} \label{sec:updates}

\subsection{Writing the book}

The first edition was written using Sweave \citep{sweave02} and \LaTeX\
\citep{lamport}, but I will
switching to either knitr \citep{knitr15} and \LaTeX\ or bookdown 
\citep{bookdown16}. For now, I will go with \emph{bookdown}.

\subsubsection{Graphics}

In the first edition of the book \citep{ehar12}, and in the current version
of the eha package \citep{eha}, the basic graphics system is used. I will 
investigate possible gains given by switching to ggplot2, lattice, or a
mixture.


\subsection{Updating wrt the R package eha}

Since the first edition was published in 2012, the eha package has
developed, and that will be picked up in the book.

\subsection{Correct possible errors and mistakes}

Readers and myself have spotted some errors and cryptic formulations over
the years. Will of course be dealt with.

\subsection{Improve examples and descriptions}

The data used in the examples of the first edition will be checked. Old
examples may be replaced by new.  

\section{Contents of the Second edition} \label{sec:contents}

\subsection*{Chapter 1: Event history and survival data}

Remains essentially as is.

\subsection*{Chapter 2: Single sample data}

Essentially as is, with the addition that Subsection 2.5. Doing it in {\bf
  R} will introduce \emph{RStudio}.

%\subsection*{Chapter 3: Reproducible research and RStudio}

% Moved to Appendix


\subsection*{Chapter 3: Cox regression}

The old Chapter 3.

\subsection*{Chapter 4: Poisson regression}

The old Chapter 4.

\subsection*{Chapter 5: More on Cox regression}

The old Chapter 5.

\subsection*{Chapter 6: Parametric models}

The old Chapter 6. The \emph{Gompertz} baseline distribution will get an
own subsection, as will the \emph{Piecewise Constant Hazard} model.

\subsection*{Chapter 7: Register-based survival data methods}

New Chapter.

\subsection*{Chapter 8: Multivariate survival models}

The old Chapter 7.

\subsection*{Chapter 9: Causality and mediation}

Expands the old Chapter 9.

\subsection*{Chapter 10: Competing risks models}

The old Chapter 8.

\subsection*{Appendices}

Essentially as is, but much have aged and will be rewritten. One important addition:
The concept of \emph{reproducible research} is
introduced, and it is shown how \emph{RStudio} can be assisting in applying
it. \emph{Rmarkdown} is introduced. These tools are used throughout the
book and will briefly be presented in an appendix.


\bibliography{surv2}

\end{document}
